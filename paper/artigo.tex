\documentclass[conference]{IEEEtran}
\usepackage{filecontents}
\usepackage[noadjust]{cite}
\usepackage[portuguese]{babel}
\usepackage[utf8]{inputenc}
\usepackage{url}
\usepackage{hyperref}
\usepackage{graphicx}
\usepackage{mathtools}
\usepackage{amsmath}

\hyphenation{op-tical net-works semi-conduc-tor}    

\title{Redes RBF em problemas de regressão}

\author{\IEEEauthorblockN{Vítor de Albuquerque Torreão}
	\IEEEauthorblockA{Departamento de Estatística e Informática\\
		Universidade Federal Rural de Pernambuco\\
		Recife, Pernambuco\\
		Email: vdat@mail.com}}


\markboth{Disciplina de Redes Neurais, Dezembro~2015}%
{Shell \MakeLowercase{\textit{et al.}}: Bare Demo of IEEEtran.cls for Journals}

\begin{document}
\maketitle

\begin{abstract}
Dentro da área de modelagem matemática, uma rede de função de base radial (RBF) 
tem como base a teoria convencional de aproximação. Uma rede RBF é uma rede 
neural artificial (RNA) que utiliza funções de base radial como 
funções de ativação dos neurônios. Essas redes constituem uma alternativa 
bastante popular às redes de Perceptron Multicamadas (MLP), por conta de sua 
estrutura mais simples e seu treinamento mais rápido. Essas redes podem ser 
utilizadas em diversos problemas de aprendizado de máquina, tanto de regressão 
como de classificação. Neste artigo, porém, será feito um estudo do potencial 
das redes RBF para resolver exclusivamente problemas de regressão. Serão 
apresentados uma implementação e um estudo de caso para validar o estudo.
\end{abstract}

\begin{IEEEkeywords}
Aprendizado de máquina, redes neurais, regressão
\end{IEEEkeywords}

\section{Introdução}


\section{Conclusão}
The conclusion goes here.

\section*{Agradecimentos}
We acknowledge the acknowledged acknowledgees.


\bibliography{artigo}
\bibliographystyle{IEEEtran}

\end{document}